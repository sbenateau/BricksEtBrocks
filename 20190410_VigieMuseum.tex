\documentclass[ignorenonframetext,]{beamer}
\setbeamertemplate{caption}[numbered]
\setbeamertemplate{caption label separator}{: }
\setbeamercolor{caption name}{fg=normal text.fg}
\beamertemplatenavigationsymbolsempty
\usepackage{lmodern}
\usepackage{amssymb,amsmath}
\usepackage{ifxetex,ifluatex}
\usepackage{fixltx2e} % provides \textsubscript
\ifnum 0\ifxetex 1\fi\ifluatex 1\fi=0 % if pdftex
  \usepackage[T1]{fontenc}
  \usepackage[utf8]{inputenc}
\else % if luatex or xelatex
  \ifxetex
    \usepackage{mathspec}
  \else
    \usepackage{fontspec}
  \fi
  \defaultfontfeatures{Ligatures=TeX,Scale=MatchLowercase}
\fi
% use upquote if available, for straight quotes in verbatim environments
\IfFileExists{upquote.sty}{\usepackage{upquote}}{}
% use microtype if available
\IfFileExists{microtype.sty}{%
\usepackage{microtype}
\UseMicrotypeSet[protrusion]{basicmath} % disable protrusion for tt fonts
}{}
\newif\ifbibliography
\hypersetup{
            pdftitle={Galaxy-Bricks},
            pdfborder={0 0 0},
            breaklinks=true}
\urlstyle{same}  % don't use monospace font for urls
\usepackage{graphicx,grffile}
\makeatletter
\def\maxwidth{\ifdim\Gin@nat@width>\linewidth\linewidth\else\Gin@nat@width\fi}
\def\maxheight{\ifdim\Gin@nat@height>\textheight0.8\textheight\else\Gin@nat@height\fi}
\makeatother
% Scale images if necessary, so that they will not overflow the page
% margins by default, and it is still possible to overwrite the defaults
% using explicit options in \includegraphics[width, height, ...]{}
\setkeys{Gin}{width=\maxwidth,height=\maxheight,keepaspectratio}

% Prevent slide breaks in the middle of a paragraph:
\widowpenalties 1 10000
\raggedbottom

\AtBeginPart{
  \let\insertpartnumber\relax
  \let\partname\relax
  \frame{\partpage}
}
\AtBeginSection{
  \ifbibliography
  \else
    \let\insertsectionnumber\relax
    \let\sectionname\relax
    \frame{\sectionpage}
  \fi
}
\AtBeginSubsection{
  \let\insertsubsectionnumber\relax
  \let\subsectionname\relax
  \frame{\subsectionpage}
}

\setlength{\parindent}{0pt}
\setlength{\parskip}{6pt plus 2pt minus 1pt}
\setlength{\emergencystretch}{3em}  % prevent overfull lines
\providecommand{\tightlist}{%
  \setlength{\itemsep}{0pt}\setlength{\parskip}{0pt}}
\setcounter{secnumdepth}{0}

\title{Galaxy-Bricks}
\subtitle{Une plateforme d'analyse de données collaborative}
\date{}

\begin{document}
\frame{\titlepage}

\begin{frame}{Itinéraire historique des données issues des sciences
participatives}

\textbf{Récolte des données} \textgreater{} Gestion des données
\textgreater{} Analyse des données \textgreater{} Résultats/Sorties
\textgreater{} Communication

\begin{figure}
\centering
\includegraphics{figures/Itineraire1.png}
\caption{Itineraire1}
\end{figure}

\end{frame}

\begin{frame}{Proposition d'un nouvel itinéraire}

\begin{figure}
\centering
\includegraphics{figures/Itineraire2.png}
\caption{Itineraire2}
\end{figure}

\end{frame}

\begin{frame}{Objectifs}

\begin{itemize}
\tightlist
\item
  Proposer de nouvelles voies pour la participation
\item
  Maintenir la participation existante
\item
  Donner aux participants les moyens de répondre aux questions qu'ils se
  posent
\end{itemize}

\end{frame}

\begin{frame}{Nécessité de concevoir des outils adaptés aux publics}

\includegraphics{figures/code_R.png}
\includegraphics{figures/outil_galaxy.png}

\end{frame}

\begin{frame}{Besoin de formation}

\begin{figure}
\centering
\includegraphics{"figures/galaxy_training_logo.png"}
\caption{galaxy\_training\_logo.png}
\end{figure}

\end{frame}

\begin{frame}{Perspectives}

\begin{itemize}
\tightlist
\item
  Analyse de donnée collaborative

  \begin{itemize}
  \tightlist
  \item
    Proposer des pistes pour l'exploration des données avec les
    participants qui contribuent directement à l'analyse
  \end{itemize}
\end{itemize}

\end{frame}

\begin{frame}{Mise en conformité par rapport aux rêgles d'accès aux
données}

\begin{itemize}
\tightlist
\item
  Obligation réglementaire
\item
  Très compliqué sans outils d'analyse
\end{itemize}

\end{frame}

\begin{frame}{}

il faut impérativement disposer d'outils d'analyses pour pouvoir ne
serait-ce que visualiser les données.

\end{frame}

\end{document}
